% Template file for an a0 portrait poster.
% Written by Graeme, 2001-03 based on his SOC poster.
%
% See discussion and documentation at
% <http://www.astro.gla.ac.uk/users/norman/docs/posters/> 
%
% $Id$



% We switch to portrait mode. This works as advertised.
\documentclass[a0,portrait]{a0poster}
% You might find the 'draft' option to a0 poster useful if you have
% lots of graphics, because they can take some time to process and
% display. (\documentclass[a0,draft]{a0poster})

% Switch off page numbers on a poster, obviously, and section numbers too.
\pagestyle{empty}
\setcounter{secnumdepth}{0}

% The textpos package is necessary to position textblocks at arbitary 
% places on the page.
\usepackage[absolute]{textpos}

% Graphics to include graphics. Times is nice on posters, but you
% might want to switch it off and go for CMR fonts.
\usepackage{graphics,wrapfig,times}

% These colours are tried and tested for titles and headers. Don't
% over use color!
\usepackage{color}
\definecolor{DarkBlue}{rgb}{0.1,0.1,0.5}
\definecolor{Red}{rgb}{0.9,0.0,0.1}

% see documentation for a0poster class for the size options here
\let\Textsize\normalsize
\def\Head#1{\noindent\hbox to \hsize{\hfil{\LARGE\color{DarkBlue} #1}}\bigskip}
\def\LHead#1{\noindent{\LARGE\color{DarkBlue} #1}\smallskip}
\def\Subhead#1{\noindent{\large\color{DarkBlue} #1}}
\def\Title#1{\noindent{\VeryHuge\color{Red} #1}}

% Set up the grid
%
% Note that [40mm,40mm] is the margin round the edge of the page --
% it is _not_ the grid size. That is always defined as 
% PAGE_WIDTH/HGRID and PAGE_HEIGHT/VGRID. In this case we use
% 15 x 25. This gives us a wide central column for text (7 grid
% spacings) and two narrow columns (3 each) at each side for 
% pictures, separated by 1 grid spacing.
%
% Note however that texblocks can be positioned fractionally as well,
% so really any convenient grid size can be used.
%
\TPGrid[40mm,40mm]{15}{25}  % 3 - 1 - 7 - 1 - 3 Columns

% Mess with these as you like
\parindent=0pt
%\parindent=1cm
\parskip=0.5\baselineskip

% abbreviations
\newcommand{\ddd}{\,\mathrm{d}}

\begin{document}

% Understanding textblocks is the key to being able to do a poster in
% LaTeX. In
%
%    \begin{textblock}{wid}(x,y)
%    ...
%    \end{textblock}
%
% the first argument gives the block width in units of the grid
% cells specified above in \TPGrid; the second gives the (x,y)
% position on the grid, with the y axis pointing down.

% You will have to do a lot of previewing to get everything in the 
% right place.

% This gives good title positioning for a portrait poster.
% Watch out for hyphenation in titles - LaTeX will do it
% but it looks awful.
\begin{textblock}{12}(0,0)
\baselineskip=3\baselineskip \Title{Test Poster in Portrait
  Format With a\\Second title line below}
\end{textblock}

\begin{textblock}{12}(0,1.5)
\LHead{Miao YU$^1$\hspace{5cm} \texttt{yufree@live.cn}\\
\hfil\break
\textsl{1:State Key Laboratory of Environmental Chemistry and Ecotoxicology,RCEES, CAS\\}}
\end{textblock}

% Put the GU logo in the top right.
\begin{textblock}{2}(13,0)
\resizebox{2\TPHorizModule}{!}{\includegraphics{./figure/institution-logo-rcees.jpg}}
\end{textblock}


% An example text block, to get you started!
\begin{textblock}{7}(0,4)
  \LHead{Introduction}
  
  Blah, blah, blah.

  We three sit in the grid.\\
  Like spades hanging in the wind.\\
  Falling down.

  Concrete poetry is a concrete poetry is a concrete poetry is a 
  concrete poetry is a concrete poetry is a concrete poetry is a 
  concrete poetry is a concrete poetry is a concrete poetry is a 
  concrete poetry is a concrete poetry is a concrete poetry is a 
  concrete poetry is a concrete poetry is a concrete poetry is a 
  concrete poetry is a concrete poetry is a concrete poetry is a 
  concrete poetry is a concrete poetry is a concrete poetry is a 
  CON

  \[
  x' = \frac{1}{x}
  \]

  Circular arguments go round and round, they don't go down the page
  like this, hence this is not a circular arguments go round and round, they don't go down the page
  like this, hence this is not a circular arguments go round and round, they don't go down the page
  like this, hence this is not a circular arguments go round and round, they don't go down the page
  like this, hence this is not a circular arguments go round and round, they don't go down the page
  like this, hence this is not a circular arguments go round and round, they don't go down the page
  like this, hence this is not a circular arguments go round and round, they don't go down the page
  like this, hence this is not a circular arguments go round and round, they don't go down the page
  like this, hence this is not a 

\end{textblock}


% Another text block in the bottom right.
\begin{textblock}{7}(8,14)
  \LHead{Introduction}
  
  Blah, blah, blah.

  We three sit in the grid.\\
  Like spades hanging in the wind.\\
  Falling down.

  Concrete poetry is a concrete poetry is a concrete poetry is a 
  concrete poetry is a concrete poetry is a concrete poetry is a 
  concrete poetry is a concrete poetry is a concrete poetry is a 
  concrete poetry is a concrete poetry is a concrete poetry is a 
  concrete poetry is a concrete poetry is a concrete poetry is a 
  concrete poetry is a concrete poetry is a concrete poetry is a 
  concrete poetry is a concrete poetry is a concrete poetry is a 
  CON

  \[
  x' = \frac{1}{x}
  \]

  Circular arguments go round and round, they don't go down the page
  like this, hence this is not a circular arguments go round and round, they don't go down the page
  like this, hence this is not a circular arguments go round and round, they don't go down the page
  like this, hence this is not a circular arguments go round and round, they don't go down the page
  like this, hence this is not a circular arguments go round and round, they don't go down the page
  like this, hence this is not a circular arguments go round and round, they don't go down the page
  like this, hence this is not a circular arguments go round and round, they don't go down the page
  like this, hence this is not a circular arguments go round and round, they don't go down the page
  like this, hence this is not a 

\end{textblock}



% If you want to add a figure do something like this:

%\begin{textblock}{3}(0,15)
%  \begin{center}
%\resizebox{3\TPHorizModule}{!}{\includegraphics{my_figure.eps}}
%\\Figure 5: Googles per Snark (renormalised with wild angry men
%  \end{center}
%\end{textblock}

\begin{textblock}{9}(3.5,21.2)
\LHead{Conclusions}

It is possible to do a portrait poster!

\end{textblock}



% Place the group logo at the bottom left - visually this balances
% well with the University logo at the top right. 
\begin{textblock}{3}(0.1,22)
  \begin{center}
\resizebox{1.5\TPHorizModule}{!}{\includegraphics{./figure/lab-logo.png}}
\color{red}\\Environmental Analysis and Toxicology\\State Key Laboratory of Environmental Chemistry and Ecotoxicology\\RCEES, CAS
  \end{center}
\end{textblock}

\end{document}

